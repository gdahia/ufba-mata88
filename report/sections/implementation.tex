\documentclass[../main.tex]{subfiles}
 
\begin{document}

O sistema foi implementado usando a linguagem Java, pois é uma linguagem amplamente conhecida e utilizada comercialmente, e que todos os integrantes do grupo possuem um bom conhecimento sobre. Além disso Java facilita a programação de sistemas distribuídos pois possui API's específicas para tais projetos, como o RMI, que foi utilizado neste trabalho. Outros aspectos como transparência, possível mobilidade e extensibilidade também foram levados em consideração e favorecendo a escolha da linguagem Java.

O RMI é uma interface que permite que metódos de objetos remotos, isto é, instâncias de objetos que não necessariamente estão na mesma máquina, possam ser chamados como uma chamada local, tornando a implementação mais simples. Além disso, o RMI proporciona uma transparência maior ao sistema, visto que a parte de comunicação entre processos é lidada por ele, ficando assim fora do alcance do usuário. O RMI também proporciona mobilidade, uma vez que objetos e código podem ser transportados pelo sistema.

\end{document}

