\documentclass[../main.tex]{subfiles}
 
\begin{document}

\subsection{Linguagem Java}

O sistema foi implementado em linguagem Java, pois esta era, dentre as analisadas, a única sobre a qual todos os integrantes do grupo tinham domínio prévio suficiente, e que também possui integração e suporte nativo a ferramentas dedicadas a construção de sistemas distribuídos.

Dentre as ferramentas nativas de Java, a mais importante para o projeto foi Java RMI, pois permitiu que o sistema fosse implementado sem que fosse necessário se preocupar com gerenciamento de memória dentre os processos e, excetuando a inicialização requerida pelo sistema, a declaração de heranças e implementações relativas ao comportamento remoto das classes e o tratamento de exceções, as chamadas pudessem ser feitas como seriam localmente, ignorando na maior parte do tempo preocupações com problemas como \textit{marshalling} e \textit{unmarshalling}, por exemplo.
A importância de Java RMI é tal que dois processos que se comunicam diretamente só o fazem através de invocações remotas de métodos.

Além de Java RMI, a simplicidade para o tratamento de ações concorrentes em Java, também uma funcionalidade nativa da linguagem, tornou a implementação do trabalho muito mais fácil do que seria em alguma outra linguagem não tão apta a tarefa.
A versão anterior do projeto utilizava mecanismos de Java para o tratamento de concorrência (o uso da classe \textit{Vector} ao invés da mais comumente utilizada \textit{ArrayList}, por exemplo) que são pouco recomendados pela comunidade por falhas.
Isso foi corrigido com utilização de métodos \textit{synchronized} nos locais apropriados e problemas de concorrência não devem mais ser observados.

\subsection{Conexão entre clientes e servidor}

Um servidor para o sistema de troca de mensagens desta implementação deve implementar a interface remota \textit{Server}.
A única implementação fornecida é a classe \textit{ServerImpl}, a principal classe do lado do servidor da aplicação.
É através da obtenção de uma referência a um objeto remoto que implementa a interface \textit{Server} (na prática, uma instância de \textit{ServerImpl}) que se dá a comunicação entre clientes e servidor.
A aplicação do servidor, implementada na classe \textit{ServerApp}, instancia um objeto da classe \textit{ServerImpl}, o exporta e faz o \textit{bind} no registro.
A invocação da aplicação do servidor deve fornecer o endereço IP (com o comando \textit{java ServerApp [endereço IP]}) ao qual clientes deverão se conectar; caso contrário, é assumido que se trata de \textit{localhost}.

A aplicação do cliente, implementada na classe \textit{ClientApp}, é responsável por obter a referência ao objeto remoto que implementa o servidor, dado o endereço IP como argumento da invocação, e instanciar a classe \textit{Client}, responsável pela interação com o usuário, com a instância obtida como argumento.

\subsection{Interação com o usuário}

A interação com o usuário se dá através de um menu textual impresso na linha de comando de um terminal.
Não há interface gráfica.
Os comandos podem ser de dois tipos: ou correspondem a numeração especificada em algum menu ou a comandos similares aos utilizados em editores de texto modais, como \textit{vi}.

Os menus são implementados nas classes de sufixo \textit{Menu} e na classe \textit{Client}.
Existe interação mínima com o usuário também nas classes com sufixo \textit{Handler}.

\subsection{Cadastro de usuários} \label{subsec:signup}

Para se cadastrar no sistema, um novo usuário deve selecionar a opção \textit{Sign up} no menu principal.
Isso ocasiona a instanciação de um objeto da classe \textit{SessionHandler}, com argumento a referência ao servidor, cujo construtor é responsável por obter o nome que o usuário deseja utilizar no sistema e suas credenciais.
Em seguida, a invocação do método \textit{signUp} da classe \textit{SessionHandler} faz com que o nome de usuário escolhido e as credenciais (ambas classes serializáveis) sejam enviadas ao servidor através de uma invocação remota do método \textit{addUser}.
Este método verifica a disponibilidade do nome de usuário requisitado -- nomes de usuário são identificadores únicos para usuários e, portanto, não podem se repetir no servidor -- e, caso esteja livre, salva as credenciais do usuário associando-as a seu nome, além da instanciação de uma sessão de usuário da classe \textit{SessionImpl}, implementação da interface \textit{Session}.
O usuário é informado com uma mensagem, na saída padrão, do sucesso ou não de sua requisição.

\subsection{Login}

Para efetuar o login no sistem, o usuário deve selecionar a opção \textit{Sign in} no menu principal.
Isso ocasiona a instanciação de um objeto da classe \textit{SessionHandler}, análogo à vista na seção \ref{subsec:signup}.
Feito isso, o servidor irá verificar a autenticidade das credenciais fornecidas utilizando as credenciais armazenadas anteriormente e, assim, a identidade do usuário.
Caso se comprove sua autenticidade, o usuário será recebido com uma mensagem de boas vindas e será redirecionado para o menu de sessão.
A nível de implementação, o login bem-sucedido termina com o servidor enviando ao cliente uma referência ao objeto remoto \textit{Session} correspondente à sessão do usuário agora logado.
Esta referência é argumento na instanciação da classe \textit{SessionMenu}, responsável pela gerência da sessão do usuário pelo processo cliente.
Caso o usuário não seja quem ele diz ser ou o nome de usuário fornecido não esteja associado a um usuário cadastrado, ele será notificado com uma mensagem de erro.

\subsection{Iniciando uma nova conversa}

Para criar novas conversas, o usuário deve selecionar a opção \textit{New Chat} no menu de sessão.
Conversas são sempre iniciadas apenas com o usuário que a criou, mas a composição de seus membros pode ser modificada com a adição de novos membros pelos membros que pertenciam a conversa ou com a saída de membros antigos.
Conversas vazias são automaticamente removidas pelo coletor de lixo do JVM, já que não possuem referências apontando para si.
No momento de sua criação, conversas recebem um tópico genérico, que pode posteriormente ser modificado.

É interessante pensar em conversas como implementações de sistemas do tipo \textit{Publish/Subscribe}: a diretiva \textit{subscribe} corresponde a entrada de um usuário em uma conversa, o ato de \textit{unsubscribe} de um processo corresponde a saída de um usuário de uma conversa, e o envio de uma mensagem para os membros da conversa seria equivalente a \textit{publish}.
Nesta visão, cada sessão de usuário seria um processo, ao invés de cada computador conectado ao servidor.
É importante notar que esses números nem sempre correspondem um-a-um: não há impedimento para que um usuário faça login em mais de um computador ao mesmo tempo.
Esta análise captura a essência da comunicação indireta desacoplada do tempo que ocorre entre os usuários quando estes se comunicam não necessariamente online ao mesmo tempo.

Uma vez selecionada a opção de criar uma nova conversa no menu, a sessão de usuário manda uma requisição para o servidor, que caso suceda, irá instanciar um objeto da classe \textit{ChatImpl}, passando como argumentos uma referência ao servidor e o nome do usuário que criou o chat.
A classe \textit{ChatImpl} implementa a interface \textit{Chat}, correspondente a uma sessão de conversa, análogo ao papel da interface \textit{Session} em relação a sessão de usuário.
Esta classe também é responsável por armazenar as mensagens trocadas na conversa em um \textit{ArrayList}.
Em seguida, essa instância é adicionada ao vetor de conversas da sessão que fez a requisição.
Caso contrário, será exibida uma mensagem de erro.

\subsection{Trocando mensagens}

Caso o usuário faça parte de alguma conversa, o menu de sua sessão a exibirá antes das opções disponíveis para usuários que não participam de conversas.
Selecionar a opção correspondente a uma conversa ocasiona a instanciação de duas classes: \textit{ChatHandler} e \textit{ChatMenu}.
\textit{ChatMenu} trata da interação com o usuário, enquanto \textit{ChatHandler} executa as requisições feitas pelo cliente, se utilizando de invocações remotas de uma instância de \textit{Chat}.

O exemplo mais claro e importante da interação entre \textit{ChatHandler} e \textit{Chat} se dá no envio de mensagens para a conversa.
Uma vez escolhida a opção de enviar uma mensagem no menu de conversa, \textit{ChatHandler} obtém do usuário, através da entrada padrão, o conteúdo da mensagem que ele deseja enviar.
Em seguida, é instanciado um objeto da classe \textit{Message} com o texto da mensagem e seu autor.
Esse objeto é enviado para a conversa através da invocação remota do método \textit{sendMessage} da interface \textit{Chat}.

A visualização das mensagens trocadas na conversa até agora é feita através de navegação.
A classe \textit{ChatMenu} mantém um atributo chamado \textit{messageIndex}, responsável por indicar qual mensagem deve ser vista em relação a mensagem mais recentemente recebida na conversa.
Não há necessidade de preocupação com a ordenação das mensagens, pois o servidor é centralizado e todas as mensagens são guardadas em um único \textit{ArrayList} em \textit{ChatImpl}, \textit{messages}.
`j' avança para a mensagem mais recente (decrementando \textit{messageIndex}), caso haja, e `k' retrocede para a mensagem menos recente; `q' retorna para o menu da conversa.

\subsection{Deletando um cadastro}

O usuário pode a qualquer momento deletar seu próprio cadastro.
Para fazer isso, ele deve selecionar a opção \textit{Delete account} no menu de sessão.
Uma vez feito isso, será exibida uma pergunta para confirmar que o usuário deseja realmente deletar sua conta, já que isso implicaria sair de todas as conversas que o usuário participa e é um procedimento irreversível.
Caso a primeira letra digitada pelo usuário seja `y', a sua conta será deletada permanentemente.
Caso contrário o usuário será redirecionado para o menu de sessão novamente.

Uma vez que o usuário seleciona a deleção de sua conta e confirma, a sessão irá chamar o método \textit{delete} que irá chamar o método \textit{removeUser} na instância do servidor que está sendo usada por aquele usuário e que recebe como argumento o nome de usuário.
O servidor irá então remover a sessão e a credencial do usuário de suas respectivas \textit{HashMap}s e ao retornar, a função \textit{delete} irá remover o usuário de todas as conversas das quais ele participava.
Feito isso o usuário irá receber uma mensagem garantindo que sua conta foi deletada, e será redirecionado para o menu principal da aplicação do cliente.

\end{document}

