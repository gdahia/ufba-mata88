\documentclass[../main.tex]{subfiles}
 
\begin{document}

\subsection{Linguagem Java}

O sistema foi implementado usando a linguagem Java, pois esta era, dentre as analisadas, a única que todos os integrantes do grupo possuíam previamente um bom conhecimento sobre, possui integração e suporte nativo a ferramentas dedicadas a construção de sistemas distribuídos.

Dentre as ferramentas nativas de Java, a mais importante para o projeto foi Java RMI, pois permitiu que o sistema fosse implementado sem que fosse necessário se preocupar com gerenciamento de memória dentre os processos e, excetuando a inicialização requerida pelo sistema, a declaração de heranças e implementações relativas ao comportamento remoto das classes e o tratamento de exceções, as chamadas pudessem ser feitas como seriam localmente.

Além de Java RMI, a simplicidade para o tratamento de ações concorrentes em Java, também uma funcionalidade nativa da linguagem, tornou a implementação do trabalho muito mais fácil do que seria em alguma outra linguagem não tão apta a tarefa.

\end{document}

