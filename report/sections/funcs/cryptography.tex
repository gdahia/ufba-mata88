\documentclass[../main.tex]{subfiles}
 
\begin{document}

O sistema garante a segurança de duas funcionalidades essenciais através de protocolos criptográficos: o login e a troca de mensagens.
Para isso, ele utiliza de implementações disponíveis na linguagem Java de geradores de número pseudoaleatório criptograficamente seguros e de criptografia com chave simétrica e com chave assimétrica.

\subsubsection{Introdução}

Seguem definições sucintas e informais dos conceitos utilizados nas subseções seguintes.

Um gerador de número pseudoaleatório criptograficamente seguro é um gerador de número pseudoaleatório com propriedades que impedem um adversário de utilizar seu funcionamento correto para invadir um sistema que, desconsiderando seu uso, seja seguro.

Um esquema de criptografia é dito assimétrico se possui um par de chaves criptográficas com a seguinte propriedade: mantendo o sigilo de uma das chaves, um processo pode receber mensagens de maneira segura através de um canal inseguro.
Assim, dois processos que troquem chaves públicas podem trocar mensagens através de um canal ao qual um adversário possui acesso.

Um chave criptográfica é dita simétrica se sua posse é necessária e suficiente para leitura do conteúdo por ela criptografado.

\subsubsection{Login}

Quando um usuário se cadastra no sistema, ele envia ao servidor uma cópia de sua chave pública.
O servidor a armazena como credencial associada a esse usuário.

Quando o usuário quer fazer login, ele requer ao servidor um código de verificação.
Isso faz com que o servidor gere, de maneira criptograficamente segura, um número aleatório grande.
É importante que essas propriedades sejam atendidas: o tamanho do número garante que soluções só possam ser encontradas via força-bruta, a aleatoriedade do número impede que encontrar a solução para poucos casos através de força-bruta não seja o suficiente para garantir acesso, e o servidor precisa gerar o número para que ele seja garantidamente grande e aleatório.
Em seguida, o servidor criptografa o número com a chave pública associada ao usuário alegante e a envia ao cliente.
Além disso, é armazenada uma cópia (substituindo possivelmente uma previamente existente) associada ao nome desse usuário.

O cliente genuíno recebe o número criptografado, o descriptografa utilizando sua chave privada, criptografa-o utilizando a chave pública do servidor e o reenvia para servidor.
O servidor descriptografa o número e compara com aquele que havia armazenado.
Caso sejam iguais, sabe que o usuário é capaz de descriptografar a chave pública do usuário e, logo, possui a chave privada.
Isso é suficiente para garantir seu acesso, já que supõe-se que apenas o usuário legítimo possui acesso a chave privada.
A cópia do número guardada no servidor é, então, descartada, independente do sucesso da tentativa de login.

A última mensagem trocada entre cliente e servidor, contudo, é problemática: o servidor envia uma referência a um objeto remoto de maneira insegura.
Um adversário observando as mensagens ganharia acesso imediato a sessão do usuário, mesmo sendo ilegítimo.
Utilizar RMI através de sockets com SSL seria suficiente para contornar esse problema, mas, como esse é um detalhe de implementação, foi ignorado na implementação do sistema.

\subsubsection{Troca de mensagens}

Como cada usuário da conversa possui uma chave pública na posse do servidor, seria suficiente criptografar uma cópia de cada uma das mensagens enviadas para cada um dos membros utilizando sua chave pública encontrada no servidor.
Isso, contudo, é indesejável por dois motivos: se uma conversa conta com $N$ participantes, seriam necessárias $N$ cópias de cada uma das mensagens; além disso, criptografia de chave pública demanda tempo considerável, apesar de fazível, fazendo com que $N$ execuções se tornassem um fardo considerável.

Optamos, então, por uma alternativa mais elegante que também permite o desacoplamento temporal dos usuários.
O criador da conversa fornece uma chave simétrica ao servidor, chamemos ela de chave da conversa, criptografada com sua própria chave pública.
Isso garante acesso ao criador da conversa à mesma e uma base indutiva para nosso protocolo.
Sempre que Alice é adicionada a uma conversa, podemos supor que Bob, o usuário que a adicionou, possui acesso a essa conversa.
Por indução, existe uma cópia da chave da conversa criptografada com a chave pública de Bob.
Bob adquire essa cópia da chave e, utilizando sua chave privada, a descriptografa, criptografa novamente utilizando dessa vez a chave pública de Alice, disponível no servidor, e a coloca na conversa.
Quando Alice estiver online, ela pode descriptograr a sua cópia da chave da conversa disponível na mesma e ganhar, assim, acesso à conversa.

Como visto, o servidor não possui acesso nem a chave privada dos usuários e nem a chave da conversa (porque as cópias lá armazenadas estão criptografadas).
Logo, o protocolo descrito acima é suficiente para que a troca de mensagens seja criptografada de ponta-a-ponta, e utilizando apenas o número usual de mensagens e $N$ cópias da chave da conversa.

Há, contudo, duas falhas na segurança do protocolo.
A primeira é que, na implementação atual, a saída de um usuário de uma conversa não ocasiona a geração de uma nova chave criptográfica: caso ele viesse a adquirir acesso ao canal de comunicação, poderia usar sua cópia da chave da conversa para ler o conteúdo das mensagens, mesmo que essas fossem trocadas após sua saída.
A segunda é que a adição de usuários a conversas não possui gerenciamento seguro: o servidor poderia fornecer sua chave pública como a chave de um usuário, sob a demanda de adição de um outro usuário qualquer, e assim ganhar acesso a chave da conversa.

\end{document}
