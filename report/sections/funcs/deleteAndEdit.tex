\documentclass[../main.tex]{subfiles}
 
\begin{document}

O usuário pode deletar ou editar mensagens que ele tenha enviado para alguma conversa, para isso ele precisa abrir o message log e navegar até encontrar a mensagem que ele deseja editar ou deletar. Ao chegar na mesagem, o usuário deve apertar o caractére 'd' e depois apertar enter. Isso irá deletar permanentemente aquela mensagem da conversa, e a mudança será refletida para todos os usuários da conversa, ou seja, nenhum usuário será capaz de ver novamente aquela mensagem. Para editar o conteúdo da mensagem, o usuário deve digitar o caractére 'e' e depois apertar enter. Ao fazer isso, o usuário deve digitar o novo conteudo da mensagem, e o conteudo antigo será substituido pelo novo conteúdo digitado, e todos os usuários passarão a ver a mensagem com seu novo conteúdo. Caso o usuário tente deletar ou editar uma mensagem enviada pelo Sistema, será mostrada uma mensagem de erro dizendo que o usuário não selecionou nenhuma mensagem, pois mensagens enviadas pelo sistema não podem ser modificadas, e caso um usuário tente editar ou deletar uma mensagem de outro usuário, será mostrada uma mensagem de erro dizendo que ele não é o autor daquela mensagem, portanto não tem o direito de editar ou deletar aquela mensagem.

Para alterar o conteúdo da mensagem, basta alterar de fato o atributo Contents do objeto da classe Message, utilizando o método setContents, que irá receber o novo conteúdo digitado pelo usuário. Para deletar a mensagem, basta remover a instância do objeto do vector de mensagens do chat correspondente. As duas implementações se baseiam no messageIndex, por isso, quando o usuário seleciona o comando de edição, chamamos o método que modifica o conteúdo da mensagem para a mensagem na posição messageIndex do vector de mensagens do chat, e quando o usuário seleciona o comando de deleção, removemos a mensagem na posição messageIndex do vector de mensagens do chat. Para garantir que o usuário não irá deletar ou editar mensagens de sistema ou de um outro usuário, comparamos o nome do usuário que deseja fazer a alteração com o nome do autor da mensagem, permitindo apenas que a alteração seja feita quando ambos forem iguais.

\end{document}
