\documentclass[../main.tex]{subfiles}
 
\begin{document}

O usuário pode deletar ou editar mensagens que ele tenha enviado para alguma conversa, para isso ele precisa abrir o message log e navegar até encontrar a mensagem que ele deseja editar ou deletar. Ao chegar na mesagem, o usuário deve apertar o caractére 'd' (sem aspas) e depois apertar enter. Isso irá deletar permanentemente aquela mensagem da conversa, e a mudança será refletida para todos os usuários da conversa, ou seja, nenhum usuário será capaz de ver novamente aquela mensagem. Para editar o conteúdo da mensagem, o usuário deve digitar o caractére 'e' (sem aspas) e depois apertar enter. Ao fazer isso, o usuário deve digitar o novo conteudo da mensagem, e o conteudo antigo será substituido pelo novo conteúdo digitado, e todos os usuários passarão a ver a mensagem com seu novo conteúdo. Caso o usuário tente deletar ou editar uma mensagem enviada pelo Sistema, será mostrada uma mensagem de erro dizendo que o usuário não selecionou nenhuma mensagem, pois mensagens enviadas pelo sistema não podem ser modificadas, e caso um usuário tente editar ou deletar uma mensagem de outro usuário, será mostrada uma mensagem de erro dizendo que ele não é o autor daquela mensagem, portanto não tem o direito de editar ou deletar aquela mensagem.

Para alterar o conteúdo da mensagem, basta alterar de fato o atributo Contents da classe Message, utilizando o método setContents, que irá receber o novo conteúdo digitado pelo usuário. Para deletar a mensagem, basta remover ela do vector de mensagens do chat correspondente. As duas implementações se baseiam no fato de que ao navegar pelas mensagens, mantemos um index chamado de messageIndex que nos diz qual a posição da mensagem atual no vector de mensagens daquele chat. Por isso, quando o usuário seleciona a edição, basta chamar o método que modifica o conteúdo da mensagem para a mensagem na posição messageIndex do vector de mensagens do chat, e para deletar, basta remover a mensagem na posição messageIndex do vector de mensagens do chat. 

\end{document}
