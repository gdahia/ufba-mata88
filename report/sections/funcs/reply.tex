\documentclass[../main.tex]{subfiles}
 
\begin{document}

O usuário também pode responder mensagens do chat.
Para isso, ele deve digitar o caractere `r' e depois apertar `Enter'.
Depois, ele deve escrever uma mensagem, que será enviada como resposta à mensagem selecionada.
O usuário pode responder mensagens de qualquer usuário, inclusive de si mesmo, e também pode responder mensagens de sistema, com exceção das mensagens de início (``\textit{<<No older message>>}'') e fim de conversa (``\textit{<<At end>>}'').

A implementação da resposta a mensagens se baseia no atributo \textit{messageIndex}.
Quando o usuário seleciona o comando de resposta a uma mensagem, o metódo \textit{replyMessage} é chamado.
Ele utiliza o \textit{messageIndex} para verificar se a mensagem é ou não uma das mensagens de início ou fim, e, caso ela realmente não seja, coletar o conteúdo e o autor da mensagem selecionada, usando os métodos \textit{getContents} e \textit{getAuthor} respectivamente.
Caso a mensagem seja a mensagem de início, ela estará na posição 0.
Caso a mensagem seja a mensagem de fim, ela estará na posição \textit{numMessages}.
Por isso, quando \textit{messageIndex} é igual a 0 ou \textit{numMessages}, é exibida uma mensagem de erro, dizendo que nenhuma mensagem foi selecionada.
Caso a mensagem não seja nem a mensagem de início e nem a mensagem de fim, o usuário irá digitar o conteúdo da mensagem de resposta, e será enviada uma nova mensagem com o conteúdo digitado pelo usuário, indicando que é uma resposta à mensagem selecionada.

Quando um usuário responde a uma mensagem, e posteriormente a mensagem que foi selecionada para resposta é editada ou removida, a mensagem de resposta não será alterada, exibindo o conteúdo da mensagem no momento em que foi selecionada pelo usuário que fez a resposta.

\end{document}
