\documentclass[../main.tex]{subfiles}
 
\begin{document}

O sistema foi implementado de forma que apenas um processo, aquele que executa a aplicação do servidor, desempenha o papel de servidor, de maneira centralizada, e todos os outros processos, os que rodam aplicações de clientes, são clientes.
A comunicação entre processos se dá exclusivamente através de troca de mensagens e pode ocorrer mesmo que eles estejam em computadores diferentes;ela ocorre, entretanto, sempre entre algum cliente e o servidor: não há comunicação direta entre processos clientes.

Não há atualmente no projeto mecanismos específicos para lidar com sua escalabilidade.
Logo, ela depende tanto da capacidade computacional proporcionada ao processo servidor e da rede onde o sistema será executado, como das funcionalidades oferecidas por padrão pela implementação.

Os testes realizados com intuito de avaliar a escalabilidade do sistema foram modestos.
Em todos, o sistema funcionou como esperado, mas o maior deles foi a criação de mais de cinco processos clientes em computadores diferentes simultaneamente.

Caso a aplicação do cliente falhe, a sessão pode ser recuperada normalmente, apenas reiniciando a aplicação do cliente, já que todos os dados se encontram no servidor, porém, caso a aplicação do servidor falhe, todos os dados serão perdidos, pois o sistema é centralizado e não são realizados tratamentos de falhas, apenas detecções de exceções devido a falhas remotas.

O usuário tem acesso somente a aplicação do cliente, e por isso não tem noção de detalhes de implementação ou de arquitetura do sistema, tendo conhecimento apenas sobre as funcionalidades que são apresentadas a ele nos menus.

\end{document}
