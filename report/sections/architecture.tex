\documentclass[../main.tex]{subfiles}
 
\begin{document}

O sistema foi implementado de forma que o servidor seja distribuído; todos os processos que executam a aplicação do servidor são ditos réplicas e desempenham coletivamente o papel de servidor.
Recursos estão localizados em exatamente uma das réplicas e cada réplica sabe em qual réplica está localizado um determinado recurso.
Todos os outros processos, os que rodam aplicações de clientes, são clientes.

\subsection{Comunicação}

A comunicação entre processos se dá exclusivamente através de troca de mensagens e pode ocorrer mesmo que eles estejam em computadores diferentes.
Ela ocorre, entretanto, sempre entre algum cliente e uma réplica ou entre réplicas: não há comunicação direta entre clientes.

\subsection{Escalabilidade}

O sistema lida com escalabilidade através do número de réplicas disponíveis.
Cada réplica lida apenas com as computações relacionadas à manutenção da consistência do servidor, à distribuição das requisições para as réplicas apropriadas e às requisições relacionadas a recursos criados localmente (chamadas a partir de agora de requisições locais).

Como as computações relativas à manutenção da consistência são proporcionais apenas ao número de réplicas e a distribuição de requisições é uma tarefa simples, a exigência maior de processamento ocorre pelo atendimento de requisições locais.
Assim, se todos os clientes fizerem requisições a uma única réplica, o sistema se comporta como se possuíse um servidor centralizado e voltamos a situação do projeto anterior.

Essa situação pode ser prevenida da seguinte maneira: ao receber uma requisição para criação de algum recurso, ao invés de atender a requisição, a réplica sorteia com probabilidade uniforme uma das réplicas e delega a ela, obrigatoriamente, a criação e gerenciamento do recurso.
Dessa forma, réplicas teriam em média o mesmo número de recursos e, supondo que o número de requisições locais cresça com o número de recursos gerenciados, dividiriam igualmente o processamento das requisições.
Esse mecanismo não foi implementado, contudo.

\subsection{Persistência}

A persistência do sistema é inteiramente dependente da persistência das réplicas, como estabelecido na seção \ref{sec:model}.
Por exemplo, caso o cliente falhe, sua sessão pode ser recuperada através do reinício da aplicação do cliente.
Contudo, caso qualquer réplica falhe ou tenha sua invocação terminada, todos os recursos nela armazenados serão perdidos, e é possível que fiquem bloqueadas a criação de outros recursos em outras réplicas, pois não há mecanismo de detecção de falhas - uma réplica pode entrar em \textit{starvation} aguardando a permissão da réplica que falhou para modificar a visão do sistema.

\subsection{Transparência}

A organização do sistema está feita em arquitetura \textit{two-tier}, garantindo parte da transparência do sistema, visto que, dessa forma, o usuário só conhece a aplicação do cliente e interfaces públicas para interação com o servidor.
Além disso, o acesso ao sistema é majoritariamente transparente: apenas sua inicialização, suas falhas remotas e sua reconfiguração são visíveis ao usuário.
O sistema suporta ainda que o acesso não garanta o conhecimento das localizações físicas dos seus recursos, nem em qual réplica o recurso está localizado, e que usuários possam acessar suas contas de múltiplas localidades simultaneamente sem que seus pares saibam.

\end{document}
