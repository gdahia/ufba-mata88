\documentclass{beamer}

\usepackage[utf8]{inputenc}

\title{Practical Byzantine Fault Tolerance}
\subtitle{Apresentação para a disciplina MATA88}
\author{Gabriel Dahia, Pedro Vidal}
\date{21 de Fevereiro de 2018}
\institute{Universidade Federal da Bahia}

\begin{document}

\frame{\maketitle}

\begin{frame}
  \frametitle{Introdução}
  \framesubtitle{Problema}

  \begin{itemize}
    \item
      Aumento da dependência de serviços online;

    \item
      Erros em software decorrentes da complexidade do software;

    \item
      Ataques maliciosos e erros ocasionam falhas bizantinas.
  \end{itemize}
\end{frame}

\begin{frame}
  \frametitle{Introdução}
  \framesubtitle{Contribuição}

  Um algoritmo prático para replicação de máquina de estados que tolera falhas bizantinas:
  \begin{itemize}
    \item
      Garante \textit{liveness} e \textit{safety};

    \item
      Funciona em sistemas assíncronos*;

    \item
      Não depende de sincronia para \textit{safety};

    \item
      Descreve otimizações que permitem uso real;
      
    \item
      Performance superior à métodos anteriores.
  \end{itemize}
\end{frame}

\begin{frame}
  \frametitle{Modelo do Sistema}
  \framesubtitle{Comunicação}

  \begin{itemize}
    \item
      Sistema distribuído assíncrono e nós são conectados por uma rede;

    \item
      A rede pode:
      \begin{itemize}
        \item
          falhar em entregar mensagens;

        \item
          atrasá-las;

        \item
          duplicá-las; ou

        \item
          entregá-las fora de ordem.
      \end{itemize}
  \end{itemize}

\end{frame}

\begin{frame}
  \frametitle{Modelo do Sistema}
  \framesubtitle{Falhas}

  \begin{itemize}
    \item
      Modelo de falhas bizantinas, exceto que nós falham independentemente;

    \item
      Pode ser garantido com:
      \begin{itemize}
        \item
          implementações diferentes por nó;

        \item
          sistemas operacionais diversos; e

        \item
          senhas e administradores distintos.
      \end{itemize}
  \end{itemize}
\end{frame}

\begin{frame}
  \frametitle{Modelo do Sistema}
  \framesubtitle{Segurança}

  \begin{itemize}
    \item
      Criptografia para prevenir \textit{spoofing}, \textit{replays} e para detectar mensagens corrompidas;

    \item
      Assinaturas de chave-pública, códigos de autenticação de mensagens, e sumário de mensagens;

    \item
      Uso de técnicas que com alta probabilidade não podem ser subvertidas.
  \end{itemize}
\end{frame}

\begin{frame}
  \frametitle{Modelo do Sistema}
  \framesubtitle{Adversário}

  \begin{itemize}
      \item
        Coordena nós comprometidos;

      \item
        Atrasa comunicação;

      \item
        Atrasa nós corretos, porém não indefinidamente.
  \end{itemize}
\end{frame}

\end{document}
